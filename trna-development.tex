\chapter{Developmental stability of the \abbr{mrna}--\abbr{trna} interface}
\label{sec:trna}

To study how changes in \mrna gene expression relate to changes in \trna gene
expression, we collected tissue samples from six time points in mouse (\mmu)
development: two before birth (E15.5 and E18.5, which stands for \num{15.5} and
\num{18.5} days after gestation of the oocyte, respectively); two shortly after
birth, happening around E20 (P0.5 and P4 --- \num{0.5} and \num{4} days after
birth, respectively) and two after weaning the juvenile mice (P22 and P29).

For each of these time points, tissue was collected from whole liver and whole
brain (\define[\defineword{homogenise} breaking apart and mixing the tissue in
such a way that all the cell types will be evenly distributed throughout the
sample]{homogenised}) and prepared for \rnaseq and \pol3 \chipseq in order to
assay \mrna and \trna gene expression. The tissues were chosen for their
interesting shifts in physiology during development: the liver is a homogeneous
organ predominantly made up of a single cell type --- around \num{70} per cent
hepatocytes --- and liver function changes fundamentally at birth; prenatal
liver serves mainly as a haematopoietic organ, whereas liver of post-natal mice
is primarily a metabolic organ \citep{Si-Tayeb:2010}. Brain, by contrast, is a
highly heterogeneous organ made up of many different cell types, with dynamic
changes all through development \citep{Liscovitch:2013}.

\alignwrap[10]{0.3\textwidth}{%
    \vspace{-0.5\baselineskip}
    \begin{tabular}{@{}lll@{}}
        \toprule
        & \abbr{mrna} & \abbr{trna} \\
        \midrule
        Raw counts per replicate & \(m_{ij}\) & \(t_{ij}\) \\
        Normalised counts per replicate & \(m_{ij}^*\) & \(t_{ij}^*\) \\
        Counts of merged replicates & \(m_{iy}'\) & \(t_{iy}'\) \\
        (Anti-)codon level counts & \(c_{xy}\) & \(a_{xy}\) \\
        \bottomrule
    \end{tabular}
    \tabcap{variables}
        {Summary of the matrix and subscript names.}
        {\(i\) is the index of a gene; \(j\) is the index of a
        library replicate; \(x\) is an (anti-)codon;  \(y\) is the index of a
        developmental stage.}}

Each experiment was performed in two biological replicates, which were highly
correlated (\cref{fig:trna-replicate-variability}).
\Cref{fig:trna-project-outline} summarises the experimental
procedure.\footnote{The wet-lab work of this project was performed by Bianca
Schmitt, who was also a joint first author on the manuscript. Claudia Kutter
provided guidance with the interpretation of the results and helped with the
creation of the figures.} \Cref{tab:variables} summarises the names used to
refer to data throughout this chapter.

\textfig[\todo{Remove background gradients from graphics}]
    {trna-project-outline}{body}{0.8\textwidth}
    {Sample analysis outline.}
    {Samples were collected in eight distinct time points. Of these, E9.5 and
    E12.5 were excluded from most of the subsequent analysis, except where
    noted. For each time point, tissue was collected from liver and brain, and
    on the one hand prepared for \rnaseq, and on the other hand cross-linked to
    the \pol3 antibody and prepared for \chipseq. The resulting data was used to
    quantify \mrna and \trna gene expression, codon usage and \trna anticodon
    abundance.}

\textfig{trna-replicate-variability}{body}{0.5\textwidth}
    {Replicate variability.}
    {Shown are the Spearman correlation coefficients between pairwise biological
    replicates for the \trna count data (left) and the \mrna count data (right).}

In addition to the six time points described above, tissue was also collected at
two earlier stages, E9.5 and E12.5. However, the embryo at such early stages of
development is too small, and the tissue development has not progressed far
enough, to permit collecting enough tissue-specific material. For that reason we
used the whole embryo at E9.5 and separated the E12.5 embryo into torso and
upper body. The subsequent analysis was performed on the six later stages in
liver and brain. However, the earlier stages confirmed the general patterns
found by analysing the remaining data (\cref{sec:appendix}).

The analysis and the results presented in this chapter are published as
\texttitle{High-resolution mapping of transcriptional dynamics across tissue
development reveals a stable \mrna[–]\trna interface} [Schmitt,
Rudolph\andothersdelim\& al., \cite*{Schmitt:2014}].

\section{Protein-coding gene expression changes dynamically during mouse
development}

To investigate protein-coding gene expression changes during development, we
quantified the \mrna abundance from \rrna-depleted \rnaseq data (strand-specific
\SI{75}{bp} paired-end reads from \name{Illumina} \name{HiSeq~2000}). Reads were
mapped to the \mmu reference genome (\abbr{ncbim37}) using
\name{\lowsc{i}\abbrsc{RAP}} \citep{Fonseca:2014} and \name{TopHat2}
\citep{Kim:2013}. Read counts were quantified using \name{HTSeq}
\citep{Anders:2014}, and assigned to protein-coding genes from the
\name{Ensembl} release \num{67} \citep{Flicek:2014}.

We excluded mitochondrial chromosomes from the analysis, because mitochondrial
genes use a slightly distinct genetic code \citep{Osawa:1989}. Furthermore, we
excluded sex chromosomes.

Changes in the expression of protein-coding genes, leading to changes in
abundance of proteins, are known to drive cellular behaviour
\citep{Brawand:2011}. Our data confirms that tissue development in mice is
accompanied by large-scale changes in the \mrna transcriptome. This is nicely
illustrated by looking at individual gene expression counts, plotted against
their genomic location (\cref{fig:mrna-expression-change}). For example, in the
liver we can confirm the functional relevance of apolipoprotein B
(\protein{APOB}) as the primary carrier of lipoproteins, which becomes
increasingly relevant as the organ shifts to metabolism \citep{Knott:1986}.
Similarly, \mrna gene expression changes highlight the role of α-fetoproteine
(\protein{AFP}) as the fetal version of serum albumin (\cref{fig:apob-afp})
\citep{Chen:1997}. In the brain, shifts can be seen in the activity of the
neural \tf \protein{FOXP2}, the expression of which continuously decreases, and
in calmodulin (\protein{CALM1}), where transcription increases after birth
\citep{Tsui:2013,Huang:2011}.

\textfloat{mrna-expression-change}{spill}{%
    \centering
    \begingroup
        \includegraphics[width=\textwidth]{liver-mrna-expression-change}
        \vspace{-\baselineskip}
        \subcaption{\label{fig:apob-afp}Gene expression changes of
            \gene{mmu}{Apob} and \gene{mmu}{Afp}.}
        \vspace{\baselineskip}
    \endgroup
    \par
    \begingroup
        \includegraphics[width=\textwidth]{brain-mrna-expression-change}
        \vspace{-\baselineskip}
        \subcaption{Gene expression changes of \gene{mmu}{Foxp2} and
            \gene{mmu}{Calm1}.}
    \endgroup}
    {Example of gene expression changes in development.}
    {The four genes are representative for tissue- and stage-specific genes,
    whose expression changes drive cell function. These changes can go up or
    down over the course of development, correspoding to either an up- or
    downregulation.}

For a more systematic analysis, I took the matrix of normalised count data of
all library replicates and all \mrna[s], \(m_{ij}^*\), for each \mrna gene \(i\)
and each library replicate \(j\), and calculated the pairwise Spearman rank
correlation between all replicates,

\begin{equation}
    cor_{ij} = \operatorname{cor}(m_{\cdot i}^*, m_{\cdot j}^*) \text{\ for all
        libraries \(i\), \(j\),}
\end{equation}

where \(m_{\cdot i}^*\) is the \(i\)th column of the matrix \(M^*\).

I then performed \pca on the correlation matrix, which allows variation in the
data to be projected onto uncorrelated axes, so that the first axis represents
the component that explains the most variance of the data, and the second axis
represents the second component.

The resulting \pca in \cref{fig:mrna-pca} shows that the biggest source of
variance in the correlation structure of the expression data is tissue identity,
which explains \num{97} per cent of the total variance. However, of the
remaining variance, \num{60} per cent is explained by progression of tissue
development in a way that nicely mirrors the known biology: the plot’s \(y\)
axis shows the linear progression of development from early stages at the bottom
to late stages at the top. We observe a much stronger variation on the \(y\)
axis for liver data: this could be explained by noting that the liver is more
homogeneous than the brain, and changes in gene expression are therefore more
coordinated; it might also reflect the change in liver function from a
haematopoietic to a metabolic organ around birth.

Next, I used \name{DESeq2} \citep{Love:2014} to identify differentially
expressed genes between stages and tissues. Genes are counted as differentially
expressed if their Benjamini–Hochberg \fdr-corrected \(p\)-value is below
\num{0.001}. The number of differentially expressed genes between all pairwise
developmental stages unsurprisingly shows that more distinct developmental
stages have higher numbers of differentially expressed genes
(\cref{fig:mrna-de-matrix}). Furthermore, there is a clear gap between pre- and
post-weaning stages, with a large jump in the number of differentially expressed
genes across the weaning boundary, in both liver and brain.

\textfigtwo{mrna-pca}
    {\pca of \mrna gene expression per developmental stage.}
    {Rotations \num{1} and \num{2} of the correlation matrix of
    protein-coding gene expression in each developmental stage. The percentage
    on the axes shows the amount of variance explained by each rotation.}
    {mrna-de-matrix}
    {Number of differentially expressed \mrna genes between stages:}
    {Each off-diagonal square shows the number of differentially
    expressed genes (at a significance threshold of \(p<0.01\)) between
    the two indicated developmental stages.}

These patterns are noteworthy because they recapitulate tissue identity and
linear progression through the stages of tissue development. But they are not
particularly surprising: cell function is dictated by the abundance of specific
proteins and thus protein-coding gene transcription. The patterns of gene
expression similarity shown in the \pca and in the number of differentially
expressed genes hence recapitulate the expected changes in cell function between
tissues and through development.

\section{Dynamic changes of \abbr{trna} gene expression during mouse
development}

Quantification of \trna genes was performed by first mapping the \pol3 \chipseq
data (non-strand-specific \SI{36}{bp} single-end reads sequenced by
\name{Illumina} \name{Genome Analyzer~IIx} or \name{HiSeq~2000}) using
\name{\abbrsc{BWA}} version 0.5.9-r16 \citep{Li:2009a} using default parameters.
Next, non-uniquely mapping reads were reallocated probabilistically according to
the description given in the previous chapter, using the \trna gene annotation
from the \name{Genomic \trna Database}, described in \citet{Chan:2009}. For each
\trna gene (again excluding mitochondrial \trna genes because the genetic code
of the mitochondrial \mrna genes differs from the nuclear genetic code), reads
were summed within each \trna gene locus and in the \SI{\pm100}{bp} flanking
regions.

\trna genes that were unexpressed in all experimental conditions were excluded
from further analysis, to reduce the effect of multiple testing
\citep{Bourgon:2010} and to exclude potential pseudogenes in the annotation. To
be called expressed, a \trna gene had to be present in all replicates of at
least one condition with a count of at least \num{10}, after size-factor
normalisation. The threshold \num{10} was chosen so that small variations in
either direction would have a minimal impact on the thresholding. The following
analysis is thus performed using \num{311} expressed out of \num{433} total
\trna genes (\num{72} per cent).

Unlike proteins, \trna[s] do not perform a cell type specific function; instead,
their continued presence is required for the maintenance of transcription in all
cellular conditions. We therefore did not expect many changes in the levels of
\trna gene expression over the course of development, and we do in fact observe
that many \trna gene expression levels remain stable (\cref{fig:trna-counts}).
Nevertheless, we also observe that around \num{50} per cent of all \trna genes
\emph{are} differentially expressed. \Cref{fig:liver-trna-expression-change}
shows a genomic locus containing \trna genes that displays these different
dynamics.

\textfigtwo{trna-counts}
    {Overview over \trna gene expression change.}
    {Bar plots show different types of \trna gene expression dynamics: \trna
    genes without change in their expression levels, \trna genes with changes to
    their expression levels, which are nevertheless expressed in all stages of
    development across both tissues; and \trna genes which are only expressed in
    a subset of all conditions.}
    {liver-trna-expression-change}
    {Example of dynamically changing \trna genes.}
    {Genomic region showing different types of \trna gene expression behaviour;
    the label colours on the x axis corresponds to the colours in
    \cref{fig:trna-counts}.}

Surprisingly, the \trna gene expression differences follow similar patterns to
those observed in \mrna genes, with the first two principal components resulting
from the application of \pca to the rank correlation matrix again corresponding
to the tissue an developmental stage (\cref{fig:trna-pca}). The observed
patterns are incompatible with mere \emph{random} expression changes (which
would result in an unordered cloud of points). Something must account for these
concerted changes in \trna gene expression.

\textfig{trna-pca}{body}{0.8\textwidth}
    {\pca of \trna gene expression per developmental stage.}
    {Rotations \num{1} and \num{2} of the correlation matrix of
    \trna gene expression in each developmental stage. The percentage on the
    axes shows the amount of variance explained by each rotation.}

In the case of the protein-coding genes, we can explain the nonrandom changes in
gene expression by known gene regulatory mechanisms, which control the
transcriptome of each cell and developmental stage. The fact that \trna gene
expression changes across development exhibit the same patterns as \mrna gene
expression changes suggests that \trna gene expression is subject to similar
regulatory constraints. We therefore attempted to explain \emph{why} \trna gene
expression requires changing in a regulated manner, and \emph{how} this \trna
gene regulation is carried out by the cell.

Our first suspicion was that changes in \mrna gene expression might lead to
changes in codon demand, since different protein-coding genes are made up from
different codons. The change in codon demand in turn could lead to a change in
anticodon supply in the form of differential \trna gene expression. This would
meet the need for efficient translation: mismatching codon and anticodon pools
would either lead to a wasteful over-production of \trna[s] of a given
anticodon, or to a bottleneck in such an anticodon supply, causing efficiency
loss in translation. Both scenarios present a suboptimal scenario for the
fitness of the cell and should reasonably be selected against. In fact, there is
some evidence that such a selection takes place
\citep{Ikemura:1981,Ikemura:1985,Yang:2008}.

We therefore went on to quantify the codon pool corresponding to each given
transcriptome, as well as the pool of available \trna genes, grouped by their
anticodon isoacceptor identity.

\section{Every mouse \abbr{mrna} transcriptome encodes the same distribution of
triplet codons and amino acids}

The codon pool of a given \mrna transcript is given by the distribution of
triplet codons in its sequence. There are \num{64} possible triplet codons, of
which \num{61} encode \num{20} different amino acids, and three encode the stop
codon, marking the end of translation.\footnote{The analysis ignores
selenocysteine, which is a \num{21}st possible amino acid, and which can be
encoded by the stop codons \codon{UGA} under rare circumstances (see
\cref{sec:trna-intro}).} Using this information and the transcript abundance
quantified by our \rnaseq data, I calculated the abundance of each triplet codon
as well as each amino acid in the transcriptome of each developmental stage. I
then compared these across stages to find out how they varied.

First, for every gene, the number of occurrences of each codon in the longest
annotated transcript (which is often called the “canonical”
transcript\footnote{\url{http://www.ensembl.org/Help/Glossary?id=346}, retrieved
2014-05-12.}) was determined and this value was multiplied by the gene’s
expression (normalised for transcript length):

\begin{equation}
    c_{xiy} = \text{codon}_{xiy} \cdot \frac{m_{iy}'}{l_{i}} \text{\ ,}
\end{equation}

where \(\text{codon}_{xiy}\) is the number of occurrences of codon \(x\) in gene
\(i\) at stage \(y\), and \(l_i\) is the length of the canonical transcript of
gene \(i\).

Next, the overall usage of each codon was obtained by summing these values
across all genes:

\begin{equation}
    c_{xy} = \sum_i c_{xiy} \text{\ .}
\end{equation}

Relative codon usages \(c_{xy}^*\) were then calculated by dividing each codon
usage by the sum of the codon usages of a given stage:

\begin{equation}
    c_{xy}^* = \frac{c_{xy}}{\sum_k c_{ky}} \text{\ .}
\end{equation}

We find that at the transcriptome level, codon abundance is highly stable across
development in both tissues (Spearman’s \(\rho > 0.97\)) ---
\cref{fig:codon-anticodon-abundance} top left shows this by way of example using
the codons for arginine in the different stages in liver.

Given this stability, we next explored how much variation should be expected, by
simulating random transcriptomes, and computing their codon usage.

I used our library-size normalised \rnaseq data to simulate background
distributions in liver and brain for each specific developmental stage. I
randomly rearranged the expression values across genes for the expressed
(“\textsc{expr}”) and all genomically annotated (“\textsc{all}”) protein-coding
genes. For each developmental stage, I created \num{100} such random background
distributions. I then calculated the triplet codon usage for the rearranged
protein-coding \rna expression distributions.

\Cref{fig:codon-anticodon-abundance} top middle and right shows that even for
simulated transcriptomes the codon usage remains unchanged. In fact, both
observed and simulated transcriptomes seem to simply reflect the codon abundance
found in the coding part of the genome, regardless of the sometimes strong
variations in gene expression between different transcriptomes.

\textfig{codon-anticodon-abundance}{spill}{\textwidth}
    {Codon and anticodon abundance across stages of development.}
    {The figure consists of three panels (right, middle, left) with three
    subfigures (top, centre, bottom) each. The left panel shows observed data
    for each of the developmental stages in liver (brain data is comparable).
    The middle and right panels show simulated data from randomised
    transcriptomes. The middle panel used only expressed genes of each
    respective stage in the simulated data, whereas the right panel uses all
    genes, also unexpressed ones). The top figures of each panel show relative
    \mrna transcript triplet codon usage, using the representative example of
    arginine. The centre figures show the relative \trna anticodon abundance of
    the arginine isotype family. The bottom right figure shows the linear
    regression of relative codon usage against relative anticodon abundance in
    liver E15.5, along with its Spearman rank correlation. Triplet codons
    without directly corresponding anticodon (grey dots) were ignored in the
    calculation. The bottom middle and right figure shows the Spearman rank
    correlation coefficient of each stage’s relative codon and anticodon
    abundance (diamond), and the range of correlation coefficients for the
    simulated codon and anticodon pools (box plots) for each stage.}

\section{Stable isoacceptor anticodon abundance through development indicates
tight regulation of \abbr{trna} gene expression}

Next, I looked at the abundance of the matching \trna isoacceptors by summing
the expression of all \trna genes belonging to the same isoacceptor family.
Anticodon abundance was calculated by averaging the expression values for all
\trna genes in a given anticodon isoacceptor family.
\Cref{fig:codon-anticodon-abundance} centre left shows the relative anticodon
isoacceptor abundance of the arginine isotype family. Again we find that the
abundance stays stable across development in both tissues (Spearman’s \(\rho >
0.96\)).

In the same way as for the \mrna transcriptome, I then simulated \num{100}
random \trna transcriptomes per developmental stage and calculated the relative
abundance of all isoacceptor families. Unlike the observed \trna data, we find
that simulated \trna transcriptomes create highly variable pools of anticodon
isoacceptors (\cref{fig:codon-anticodon-abundance} centre middle and right). The
variability observed here is explained by the fact that there are only \num{433}
\trna genes in \mmu (of which only \num{311} were expressed in our samples),
compared to the approximately \num{20000} protein-coding genes, which leads to a
bigger relative influence of random sampling on the distribution. In contrast to
\mrna gene expression, the stable anticodon isoacceptor abundance distribution
we observe across development is therefore not compatible with random variation
in the \trna gene expression: instead, it demonstrates the necessity of a
mechanism actively stabilising \trna gene expression variation at the anticodon
isoacceptor level.

\section{\abbr{mrna} triplet codon usage is highly correlated with \abbr{trna}
anticodon isoacceptor abundance during development}

Codons and anticodon-carrying \trna[s] form the biochemical interface between
the genetic code and the amino acid sequence of proteins during \mrna
translation. I investigated this correspondence between \mrna-driven codon
demand and \trna anticodon supply by looking at the correlation between a
codons’ usage in the \mrna transcriptome and its matching \trna anticodon
isoacceptor abundance at the same developmental stage
(\cref{fig:codon-anticodon-abundance} bottom left).

To compare how well the anticodon supply of a given transcriptome was adapted to
its codon demand, I initially calculated the Spearman rank correlation between
the codon usage and the anticodon isoacceptor abundance. However, since not all
codons have a corresponding anticodon-carrying \trna, unmatched “orphan” triplet
codons were discarded from the calculation. Consequently, the correlation
coefficients I calculated ignore the possibility of wobble base pairing.

It is possible to account for wobble base pairing in several different ways
(reviewed in \citet{Gingold:2011}). In particular, \citet{Dos_Reis:2003}
describe the \tai, which takes into account the possible wobble base pairings
when calculating the fit between codon usage and \trna gene copy number. For my
analysis, rather than accounting for all possible base pairings, I opted for a
simplified version in which only unmatched codons were treated differently, and
all other codons were matched directly to their corresponding anticodons, as
described above. Orphan codons were matched to all anticodons that they
recognise via wobble base pairing, by distributing the \trna abundance of these
\trna genes between all matched codons. More abundant codons received
proportionally more of the anticodon abundance. Let \(a\) be the abundance of a
single anticodon that matches codons \(1\dots n\), and \(c_1\dots c_n\) their
codon usage. Then, for the sake of correlating codon usage with anticodon
abundance, we calculated the adjusted abundance \(a^*\) as

\begin{equation}
    a_i^* = \frac{a \cdot c_i}{\sum_j c_j} \text{\ .}
\end{equation}

Afterwards, we calculate codon--anticodon correlations using the codon usage and
the adjusted anticodon abundance. This yielded broadly comparable results to the
simple correlations ignoring wobble base pairing and unmatched codons
(\cref{sec:appendix}). I will discuss how to improve this using a \tai adapted
to \trna gene expression in the conclusion (\cref{sec:conclusion}).

Across both tissues and all stages of development, we find that \mrna triplet
codon demand and \trna anticodon isoacceptor abundance are highly correlated
(\(0.64 < \rho \leq 0.76\) Spearman’s rank correlation, all \(p < 0.001\)),
ignoring wobble base pairing. Accounting for wobble base pairing in the
calculation of the adaptation of codon demand and anticodon supply does not
substantially change these numbers.

We can compare these correlations between \mrna codon demand and \trna anticodon
supply with the correlations we find between our simulated \mrna and \trna
transcriptomes. To calculate correlations for the simulated transcriptomes, I
first determined the means for each of the \num{100} shuffled triplet codon
distributions and calculated their Spearman rank correlation with each of the
\num{100} shuffled isoacceptor distributions.

In fact, correlating all \num{100} randomly simulated \trna
transcriptomes per tissue with the simulated \mrna transcriptomes yields a
distribution of significantly lower rank correlation coefficients
(\cref{fig:codon-anticodon-abundance} bottom middle and right).

This result provides further evidence that random variation of \trna gene
expression cannot account for the observed patterns of \trna gene expression,
and that \trna gene expression must be actively regulated to stabilise the
steady abundance of the of \trna anticodon isoacceptors, matching the triplet
codon demand of the corresponding \mrna transcriptome.

\section{Variable chromatin accessibility may influence \abbr{trna} gene
transcription}

Having established that \trna gene expression varies in a controlled fashion
through development, we were next interested in uncovering the mechanism driving
this variation. From what we know about the regulation of protein-coding genes,
it seemed likely that local genomic features around each \trna gene would be
implicated in its transcriptional regulation.

Previously published results indicate that there is no clear relationship
between sequence variation of the internal promoters of a \trna gene and their
expression levels \citep{Oler:2010,Canella:2012}. We therefore focussed on the
sequence upstream of the \tss of \trna genes to search for \emph{cis}-regulatory
regions.

To this end, I collected the sequence on the forward and reverse strand of the
\SI{500}{bp} upstream regions of \trna genes that were differentially expressed
between each pair of developmental stages. These sequences were cleaned of
low-complexity regions using the \name{dust} application \citep{Bailey:2009}.
Motif enrichment analysis in the sequences was conducted with \name{MEME}
\citep{Bailey:2009}, configured to search for zero or one occurrences of one
motif per sequence, up to a maximum of three distinct motifs, with a minimum
motif size of \SI{6}{bp}. A first-order Markov model built from the upstream
regions of all nondifferentially expressed \trna[s] in the appropriate
stage–stage contrast was used as background.

Subsequently, \name{TOMTOM} \citep{Gupta:2007} was used to search for motifs
enriched in the \name{MEME} output by exploiting databases of known \tf binding
sites. I used the databases \identifier{JASPAR\_CORE\_2009\_vertebrates} and
\identifier{uniprobe\_mouse}. A minimum overlap of \SI{5}{bp} with an
\(E\)-value threshold of \num{10} was required. The analysis failed to reveal
strongly enriched, annotated motifs present consistently across stages,
suggesting that the upstream region of \trna genes does not contain regulatory
sequences explaining the observed differences in \trna gene expression (table of
all hits in \cref{sec:appendix}).

In the absence of clear evidence for nearby \tf binding sites, we hypothesised
that the transcription regulation of nearby protein-coding genes might influence
\trna gene expression. I therefore went on to look for enrichment of
differentially expressed \trna genes in close vicinity to differentially
expressed protein-coding genes.

A test for colocalisation of differentially expressed \trna genes and
differentially expressed \mrna genes was performed between developmental stages
(E15.5–P22 in liver and P4–P29 in brain, because those were the contrasts with
the largest number of differentially expressed \trna genes). For each
up-regulated \trna gene \(i\) we counted the number of up-regulated
protein-coding genes, \(n_i\), and the total number of protein-coding genes,
\(b_i\), in a genomic region centred on the \trna gene of interest. The analysis
was performed for different window sizes (\SIlist{10;50;100}{kb}). This allowed
us to compute the ratio \(r_i = {n_i}/{b_i}\), which represents \todo{… xyz}. We
repeated this analysis for each non-differentially expressed \trna gene \(j\) to
obtain the ratio \(r_j^*\). A Kolmogorov–Smirnov test was performed to assess
whether the distribution of \(r\), corresponding to the ratios of up-regulated
protein-coding genes in the vicinity of up-regulated \trna genes, was
significantly different from the distribution \(r^*\) in the vicinity of
nondifferentially expressed \trna genes with varying significance thresholds
(\numlist{0.1;0.05;0.01}).

As for the case of \tf binding sites, we were unable to demonstrate such an
association.\todo{figure}

Besides \emph{cis}-regulatory \emph{sequence} features, another possibility is
that chromatin modifications are associated with the changes in \trna gene
expression that we observed. Previous studies by \citet{Barski:2010,Oler:2010}
indicate that several chromatin modifications have an influence on \pol3-driven
transcription. Using publicly available, previously published data
\citep{Shen:2012} (\geo accession \identifier{GSE29184}), I investigated three
histone modifications associated with genomic regions containing promoters and
enhancers (H3K4me3, H3K4me1, H3K27ac) as well as \pol2 and an insulator, \ctcf.
For each of these factors, I assayed their association with

\begin{shortenumerate}
    \item active versus inactive \trna genes in embryonic (E15.5) and adult
        (P29) tissues; and
    \item differentially expressed \trna genes between E15.5 and P29
\end{shortenumerate}

in both liver and brain.

To test for association of a factor with \trna gene expression, I investigated
whether a given factor was present in the vicinity of a \trna locus. I then
performed Fisher’s exact test on the number expressed versus unexpressed \trna
genes with at least one mark in its vicinity. Occurrence of these chromatin
marks was measured \SIlist{0.1;0.5;1}{kb} upstream of and downstream from \trna
genes. Our embryonic (E15.5) and adult (P29) \pol3 data was complemented with
embryonic (E14.5) and adult (P56) \chipseq data, as different time points were
selected in our and in the \citet{Shen:2012} study. Likewise, our brain P29 data
was compared with P56 data by using the union of “cortex” and “cerebellum”
\chipseq binding locations from \citet{Shen:2012}.

In liver, between E15.5 and P29, we did indeed find a significant association
between differentially expressed \trna genes and levels of H3K27ac (\(p <
10^{-4}\), Fisher’s exact test; \cref{tab:histone-results}\todo{Wrong table.
Mention others? Which?}). H3K4me3 and \pol2
showed a less strong association, and no significant association with H3K4me1 or
\ctcf was found.

\begin{table}[h!]
    \centering
    \sisetup{
        table-figures-integer=1,
        table-figures-decimal=2,
        table-figures-exponent=2,
        table-sign-exponent=true,
        table-number-alignment=right
    }
    \begin{tabular}{@{}lSS@{}}
        \toprule
        & \multicolumn{2}{@{}c@{}}{{Developmental stage}} \\
        \cmidrule(l){2-3}
        Factor & \multicolumn{1}{r}{{Embryo}} & \multicolumn{1}{r@{}}{{Adult}} \\
        \midrule
        H3K4me3 & 1.85e-32 & 1.27e-29 \\
        Enhancer & 1 & 1.46e-2 \\
        H3K27ac & 9.00e-33 & 1.99e-21 \\
        \abbr{pol2} & 9.38e-5 & 1.85e-4 \\
        \abbr{ctcf} & 2.24e-3 & 2.09e-4 \\
        \bottomrule
    \end{tabular}

    \tabcap{histone-results}
    {Evidence for enrichment of different factors near expressed \trna genes.}
    {Shown are \(p\)-values for the hypothesis of no difference in presence of
    the specified factor between expressed and unexpressed \trna genes in
    embryonic and adult tissue liver, using a window size of \SI{\pm 0.5}{kb}.}
\end{table}

Though limited, this association of differentially expressed \trna genes histone
marks indicates that the accessibility of the chromatin may have an influence on
\trna gene expression, and that the observed differences may be partially
influenced by changing histone modification status through the course of tissue
development.

\section{\abbr{trna} anticodon isoacceptor families are transcriptionally
compensated across development}

The results thus far demonstrate that \trna gene expression varies across
development, and this variation follows clear patterns, which require active
regulation. We furthermore find that variability within the transcribed
\trna pool vanishes at the isoacceptor level: the \trna genes within each
anticodon isoacceptor family vary across developmental stages, but the sum of
their expression is stable (\cref{fig:codon-anticodon-abundance}).

This might imply (anti-)correlation of expression across stages between the
genes of an isoacceptor family. Alternatively, \trna gene expression might vary
randomly without, regard to other \trna genes in the same isoacceptor family. To
test this, we compared the distribution of correlations between genes within
each isoacceptor family with a background distribution. The background was
generated by permuting the order of the stages before calculating the \trna gene
expression correlations. Importantly, these background distributions have a
unimodal shape centred on \num{0}
(\cref{fig:compensation-b,fig:compensation-d}). This allows us to test whether
the observed correlations significantly diverge from the background model:

For each isoacceptor that is encoded by more than two \trna genes, we calculated
Spearman’s rank correlation (across developmental stages) between the expression
values of each pair of its corresponding \trna genes, i.e.\ we calculate

\begin{equation}
    c_{ij} = \operatorname{cor}(x_i, x_j) \text{\ for \(i, j \in T\),
        \(i < j\),}
\end{equation}

where \(T\) is the set of \trna genes in the isoacceptor family, and \(x_i\) is
the vector of expression values of the \(i\)th \trna gene across all stages of
development. For the same set of genes, we calculated a null set of correlations
as follows:

\begin{equation}
    b_{ijk} = \operatorname{cor}(\operatorname{perm}_k(x_i), x_j)
        \text{\ for \(i, j \in T, i < j; k \in 1\dots\lvert x_i\rvert!\)\ .}
\end{equation}

Here, \(\operatorname{perm}_k(x_i)\) is the \(k\)th permutation of the vector
\(x_i\).

Next, we used the \(\chi^2\)-test to investigate whether there was a significant
difference between the background \(b\) and the observed correlation
distributions \(c\). We only performed the test for the \num{27} isoacceptor
families with six or more genes, since isoacceptor families with less than six
genes did not contain enough points for meaningful interpretation.

The distribution of observed correlations in some cases has a bimodal shape,
which can be clearly distinguished from the unimodal background
(\cref{fig:compensation-b}). In total, \num{16} out of \num{27} isoacceptor
families (\num{59} per cent) with more than five genes show significantly
different foreground and background distributions (all \fdr-corrected \(p <
0.0199\), see \cref{tab:compensation}).

The bimodal shape of the correlation distribution can be interpreted as the
existence of two distinct clusters of \trna genes within the isoacceptor
families, which compensate for each others’ expression changes. However, these
clusters of genes do not form genomic clusters, i.e.\ the \trna genes within
each cluster are not closer to one another than to other clusters.

To establish this, I defined \num{69} clusters of all genomically annotated
\trna genes that lie within \SI{7.5}{kb} of each other. I counted how many
active \trna genes of an isoacceptor family colocalised in a genomic cluster
with \trna genes of the same isoacceptor family, before calculating the fraction
of \trna for each isoacceptor family belonging to a genomic cluster. To test
whether genes in isoacceptor families tend to genomically colocalise more than
expected by chance, we randomly assigned \trna genes to isoacceptor families
(preserving the actual isoacceptor family gene numbers) \num{1000} times. I then
tested whether the mean percentage of clustering \trna genes per isoacceptor
family differed from the mean percentage expected by chance, by using a binomial
test. Finally, I tested whether there was a difference in these percentages
between isoacceptor families that show evidence for compensation, and
isoacceptor families that show no such evidence by applying a \(\chi^2\)-test.

\parrule

In summary, we show that \trna gene expression varies pervasively across mouse
development in different tissues. This variation follows concerted patterns that
are evidence of specific regulation of \trna gene transcription. Although we
have been unable to pinpoint a mechanism for the specific gene expression
patterns  we observed, there is a broad correlation between \trna gene activity
and the existence of histone marks for active gene expression. The precise
purpose of the \trna gene regulation remain similarly unclear, since we find that
both the codon usage and the anticodon isoacceptor abundance are stable
across development, and thus do not need to be adjusted using specific \trna
gene expression changes.

Indeed, when looking at individual anticodons, we find that \trna genes within
many isoacceptor families are acting in concert to compensate for changes in
each others’ expression, with the net result of producing a stable abundance of
\trna molecules of the anticodon isoacceptor. The regulatory mechanism enacting
this compensation has thus far not been described, and its identification poses
a new challenge.

\textfloat{compensation}{spill}{%
    \centering
    \begin{minipage}{0.45\textwidth}
        \centering
        \includegraphics[height=6cm]{correlation-cag}
        \subcaption{\label{fig:compensation-a}Isoacceptor \anticodon{CAG} \trna
            gene expression levels.\\\ }
    \end{minipage}
    \hspace{0.05\textwidth}%
    \begin{minipage}{0.4\textwidth}
        \centering
        \includegraphics[height=6cm]{compensation-cag}
        \subcaption{\label{fig:compensation-b}Density curve of isoacceptor
            \anticodon{CAG} \trna gene expression correlations.}
    \end{minipage}
    \par
    \begin{minipage}{0.45\textwidth}
        \centering
        \includegraphics[height=6cm]{correlation-gcc}
        \subcaption{\label{fig:compensation-c}Isoacceptor \anticodon{GCC} \trna
            gene expression levels.\\\ }
    \end{minipage}
    \hspace{0.05\textwidth}%
    \begin{minipage}{0.4\textwidth}
        \centering
        \includegraphics[height=6cm]{compensation-gcc}
        \subcaption{\label{fig:compensation-d}Density curve of isoacceptor
            \anticodon{GCC} \trna gene expression correlations.}
    \end{minipage}}
    {\trna gene expression is compensated at the anticodon isoacceptor level
    during mouse development.}
    {…}

\begin{table}[h!]
    \centering
    \sisetup{
        table-figures-integer=1,
        table-figures-decimal=2,
        table-figures-exponent=2,
        table-sign-exponent=true,
        table-number-alignment=right
    }
    \begin{tabular}{@{}l@{}S@{}}
        \toprule
        Isoacceptor & \multicolumn{1}{r@{}}{{\(p\)-value}}\\
        \midrule
        \anticodon{GTG} & 1.01E-25 \\
\anticodon{AGC} & 2.26E-12 \\
\anticodon{GCA} & 9.17E-11 \\
\anticodon{CCA} & 1.15E-10 \\
\anticodon{CTG} & 3.60E-09 \\
\anticodon{TGG} & 3.68E-07 \\
\anticodon{AAC} & 3.68E-07 \\
\anticodon{GTA} & 5.69E-06 \\
\anticodon{CAT} & 8.53E-06 \\
\anticodon{TTC} & 2.42E-05 \\
\anticodon{TGC} & 2.56E-05 \\
\anticodon{AGA} & 6.57E-05 \\
\anticodon{CTC} & 6.81E-05 \\
\anticodon{CAC} & 2.06E-04 \\
\anticodon{GTC} & 8.76E-04 \\
\anticodon{CAG} & 1.99E-02 \\
\anticodon{GTT} & 6.01E-02 \\
\anticodon{AGT} & 1.26E-01 \\
\anticodon{GCC} & 1.33E-01 \\
\anticodon{TTT} & 3.35E-01 \\
\anticodon{GCT} & 3.45E-01 \\
\anticodon{AAT} & 3.45E-01 \\
\anticodon{GAA} & 6.50E-01 \\
\anticodon{ACG} & 8.01E-01 \\
\anticodon{TCC} & 8.01E-01 \\
\anticodon{CTT} & 8.12E-01 \\
\anticodon{AGG} & 8.44E-01 \\

        \bottomrule
    \end{tabular}

    \tabcap{compensation}{Evidence against absence of compensation.}
    {The first column contains the \trna anticodon isoacceptor families. The
    second column contains the \abbr{fdr}-adjusted \(p\)-values of \(H_0\):
    there is no effect of the order of the stages on coordinated gene expression
    of \trna genes within an isoacceptor family.}
\end{table}
