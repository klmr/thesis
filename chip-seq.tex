\chapter{Pol~III \abbr{chip}-sequencing quantifies \abbr{trna} gene
expression}
\label{sec:chip}

\section{\abbr{chipseq} is a \abbr{dna} binding assay}

\chipseq is a family of assays based on high-throughput sequencing, like
\rnaseq, but predating the latter \citet{Johnson:2007}. There are several
distinct applications of \chipseq but they all rely on the identification and
quantification of binding sites of specific proteins to \dna. The most common
use of \chipseq is the identification of novel \tf binding
sites\todo[ref]{ENCODE and others}.

Briefly, sample is prepared by cross-linking proteins to the \dna in solution
using formaldehyde. Next, sonication or MNase treatment is used to shear \dna
into smaller fragments. Some of these fragments will have the protein of
interest bound. Using an antibody that recognises the protein of interest as
specifically and sensitively as possible, these fragments are purified. The
protein is then unlinked and the \dna fraction is again purified, size selected,
ligated to sequencing adapters, amplified and sequenced
(\cref{fig:chip-seq-workflow}).\todo[ref]{}\todo{Something about the challenges of ChIP (antibody affinity etc.)}

\textfig{chip-seq-workflow}{body}{0.7\textwidth}
    {Possible \chipseq workflow.}
    {\todo{Add correct image, explanation}}

\section{Quantifying expression of \abbr{trna} genes}

We quantified \trna gene expression via \pol3 \chipseq, using an antibody that
recognises the \pol3 subunit RPC1/155, which is involved in active \trna gene
transcription. The reason for using this, on the first glance, indirect measure
is because \trna genes are not identifiable by their sequence alone: performing
a multiple sequence alignment of \trna genes in \mmu reveals that several \trna
genes share the exact same sequence (\cref{fig:trna-alignment}).

\textfloat{trna-alignment}{spill}
    {\footnotesize\begingroup
\let\m\mismatch
\begin{tabular}{@{}ll@{}}
    \toprule
    chr5.trna1044 &  \seq{GTCTCTGTGGCGCAATCGGTtAGCGCGTTCGGCTGTTAACCGAAAG...........GtTGGTGGTTCGAGCCCACCCAGGGACG}\\
    chr3.trna750 &   \seq{GTCTCTGTGGCGCAATCGGTtAGCGCGTTCGGCTGTTAACCGAAAG...........GtTGGTGGTTCGAGCCCACCCAGGGACG}\\
    chr3.trna298 &   \seq{GTCTCTGTGGCGCAATCGGTtAGCGCGTTCGGCTGTTAACCGAAAG...........GtTGGTGGTTCGAGCCCACCCAGGGACG}\\
    chr3.trna294 &   \seq{GTCTCTGTGGCGCAATCGGTtAGCGCGTTCGGCTGTTAACCGAAAG...........GtTGGTGGTTCGAGCCCACCCAGGGACG}\\
    chr3.trna289 &   \seq{GTCTCTGTGGCGCAATCGGTtAGCGCGTTCGGCTGTTAACCGAAAG...........GtTGGTGGTTCGAGCCCACCCAGGGACG}\\
    chr2.trna1947 &  \seq{GTCTCTGTGGCGCAATCGGTtAGCGCGTTCGGCTGTTAACCGAAAG...........GtTGGTGGTTCGAGCCCACCCAGGGACG}\\
    chr1.trna1014 &  \seq{GTCTCTGTGGCGCAATCGGTtAGCGCGTTCGGCTGTTAACCGAAAG...........GtTGGTGGTTCGAGCCCACCCAGGGACG}\\
    chr11.trna1446 & \seq{GTCTCTGTGGCGCAATCGGTtAGCGCGTTCGGCTGTTAACCGAAAG...........GtTGGTGGTTCGAGCCCACCCAGGGACG}\\
    chr10.trna390 &  \seq{GTCTCTGTGGCGCAATCGGTtAGCGCGTTCGGCTGTTAACCGAAAG...........GtTGGTGGTTCGAGCCCACCCAGGGACG}\\
    chr3.trna757 &   \seq{GTCTC\m CGTGGCGCAAT\m CGGT\m cAGCGCGTTCGGCTGTTAACCGAAAG...........GtTGGTGGTTCGAGCCCACCC\m GGGGACG}\\
    chr3.trna283 &   \seq{GTCTCTGTGGCGCAATTGGTtAGCGCGTTCGGCTGTTAACCGAAAG...........GtTGGTGGTTC\m AAGCCCACCCAGGGACG}\\
    \bottomrule
\end{tabular}
\endgroup

    \todo{Fix alignment of table}}
    {Alignment of Asn \trna genes.}
    {Parts of a multiple sequence alignment of \trna genes in \mmu generated
    with COVE\@. Shown are the \trna genes coding for Asn. Bases which differ
    from the consensus sequence are highlighted in red.}

Consequently, to identify individual \trna genes and to quantify their
expression, we cannot use conventional \rnaseq, since the \rna reads covering
only the transcribed gene region are not uniquely mappable. Common strategies
for counting ambiguously mapping reads, as used in \name{ERANGE} by
\citet{Mortazavi:2008}, still require at least \emph{some} unambiguous
information, to distinguish different genes that share reads.

We solve this problem by extending the \trna gene body into the flanking
regions, which are not under purifying selection, and therefore not conserved.
As \pol3 \chipseq fragments cover the flanking regions as well as the actual
gene body (\cref{fig:trna-pol3-binding-profile}), we can use reads uniquely
mapping to the flanking regions to assign ambiguously mapped reads to the
appropriate regions (\cref{fig:trna-pol3-map-ambiguous-reads}).

\textfig{trna-pol3-binding-profile}{spill}{\textwidth}
    {\trna \pol3 \chip binding profile.}
    {The shaded, bell-shaped area shows an idealised binding profile of \chipseq
    data spanning the \trna gene with the A and B box highlighted, as well as
    its flanking regions upstream and downstream of the gene body. This overlap
    plays a role in identifying the individual gene.}

More precisely, when mapping the reads, we do not discard all ambiguously
mapping reads. However, we still discard reads which are likely \pcr duplicates,
i.e.\ we remove all but one copy of non-unique reads in the raw input data.
Despite this, we still have reads that have not been uniquely assigned to a
given \trna gene (\cref{fig:trna-pol3-map-a}). To assign these reads, we use the
number of uniquely mapping reads in \trna genes’ flanking regions to determine
the most likely origin (\cref{fig:trna-pol3-map-b}) \citep{Kutter:2011}.

Formally, let \(i\) be the \(i\)th \trna gene locus, and \(c_i\) be the count of
uniquely mapped reads in its flanking region (we used \SI{\pm100}{bp}, which has
been shown to work well in practice \citep{Kutter:2011}). A multi-mapping read
\(r\), which maps to a set \(T\) of candidate \trna[s], can be allocated to a
target \trna gene \(i\) randomly with probability

\begin{equation}
    p_i = \begin{cases}
        c_i\left(\sum_{x \in T}c_x\right)^{-1} &
            \text{if \(\sum_{x \in T}c_x \neq 0\),} \\
        \vert T \rvert^{-1} & \text{otherwise.}
    \end{cases}
\end{equation}

\textfloat{trna-pol3-map-ambiguous-reads}{spill}{%
    \centering
    \begingroup
        \includegraphics[width=\textwidth]{trna-pol3-map-ambiguous-reads-1}
        \subcaption{\label{fig:trna-pol3-map-a}Two potential match candidate
            \trna genes for a read.}
    \endgroup
    \begingroup
        \includegraphics[width=\textwidth]{trna-pol3-map-ambiguous-reads-2}
        \subcaption{\label{fig:trna-pol3-map-b}Using the count data from the
            flanking regions to extrapolate most likely mapping positions for
            ambiguous reads.}
    \endgroup}
    {Mapping ambiguous \chip reads.}
    {\chip reads originating from \trna genes can often not be mapped
    unambiguously to any given \trna. Instead, information form the gene’s
    flanking regions is used to determine the more likely provenance.}

Quantification of \trna genes was performed by first mapping the \pol3 \chipseq
data (non-strand-specific \SI{36}{bp} single-end reads sequenced by
\name{Illumina} \name{Genome Analyzer~IIx} or \name{HiSeq~2000}) using
\name{BWA} version 0.5.9-r16 \citep{Li:2009a} using default parameters. Next,
non-uniquely mapping reads were reallocated probabilistically according to the
description given above, using the \trna gene annotation from the \name{Genomic
\trna Database}, described in \citet{Chan:2009}. For each \trna gene (excluding
mitochondrial \trna genes, for the same reason for which we also excluded
mitochondrial \mrna genes), reads were summed within each \trna gene locus and
in the \SI{\pm100}{bp} flanking regions.

\trna genes that were unexpressed in all experimental conditions were excluded
from the further analysis, to reduce the effect of multiple testing
\citep{Bourgon:2010}, and to exclude potential pseudogenes in the annotation. To
be called expressed, a \trna gene had to be present in all replicates of at
least one condition with a count of at least \num{10}, after size-factor
normalisation. The threshold \num{10} was chosen so that small variations in
either direction would have a minimal impact on the thresholding.

%\section{Normalisation of \trna gene expression}
%
%Raw \trna gene expression counts show strong differences between sequencing
%libraries. For \mrna count data, it is customary to use \name{DESeq}’s
%\dfn{library size normalisation} \citep{Anders:2010}. However, as shown in
%\cref{fig:trna-libraries-example}, library size normalised data still follows
%distinct distributions, which we deem biologically implausible, and probably due
%to technical bias.
%
%As a consequence, we opted to \dfn{quantile normalise} the count data\todo{find
%reference for quantile normalisation}.
%
%\textfloat{trna-libraries-example}{body}
%    {\centering\Huge Placeholder}{Placeholder}{Placeholder}
