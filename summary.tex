\chapter*{Summary}

The genetic code describes how a sequence of codons on an \abbr{mrna} is
translated into a sequence of amino acids, forming a protein. The genetic code
is physically manifested in the cell as \abbr{trna} molecules, which fall into
several classes of anticodon isoacceptors, each decoding a single codon into its
corresponding amino acid. In this thesis I discuss the central importance of the
codon--anticodon interface to \abbr{mrna}-to-protein translation, and how its
stability is maintained during the life of the cell.

This thesis summarises our research into the control of the abundance of
\abbr{trna}s by \abbr{trna} gene expression in mammalian organisms. I hope to
show that \abbr{trna} gene expression is subject to tight regulation, and that
the abundance of \abbr{trna} molecules is thus kept highly stable even across
vastly different cellular conditions, in marked contrast with the abundance of
protein-coding genes, which is changing dynamically to drive cell function.
